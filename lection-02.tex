\documentclass[aspectratio=169]{beamer}
\setbeamertemplate{navigation symbols}{}
\usepackage{graphicx}
\usepackage{stmaryrd}
\usepackage[T1,T2A]{fontenc}
\usepackage[utf8]{inputenc}
\usepackage[english,russian]{babel}
\usepackage{amsmath}
\usepackage{amsfonts}
\usepackage{amssymb}
\usepackage{cmll}
\usepackage{makeidx}
\usepackage{verbatim}
\usepackage{amsthm}
\usepackage{bnf}
\usepackage{tikz}
\usepackage{enumerate}
\usepackage{mathtext}
\usepackage{mathtools}
\usepackage{mathabx}
%\usepackage[left=2cm,right=2cm,top=2cm,bottom=2cm,bindingoffset=0cm]{geometry}
\usepackage{proof}
%\usepackage{paracol}
%\usepackage{enumitem}
\usepackage{color}
\usepackage{colortbl}
\usepackage{minted}
%\usepackage{hyperref}
\usetikzlibrary{graphs}
\usetikzlibrary{graphs.standard}
\usetikzlibrary{automata,positioning}
\usepackage{float}

\begin{document}

%\theoremstyle{dfn}
\newtheorem{dfn}{Определение}[section]
\newtheorem{nte}{Замечание}[section]

\newtheorem{axiom}{Аксиома}[section]
\newtheorem{thm}{Теорема}[section]
\newtheorem{lmm}[theorem]{Лемма}
\newtheorem{statement}{Утверждение}[section]
\newtheorem{oun_paragraph}{Пункт}[section]
\newtheorem{cons}{Следствие}[section]
\newtheorem*{exm}{Пример}

\newcommand{\comb}[1]{\operatorname{\bf{\textrm{#1}}}}
\newcommand{\func}[1]{\operatorname{#1}}
\newcommand{\reduction}[1]{{\color{OrangeRed}#1}}
\newcommand{\set}[1]{\left\{#1\right\}}

\def\from#1{\par \parbox{0.7\textwidth}{\par \hfill\raggedleft \it #1}} 

\begin{frame}{}
\begin{center}\Large Лекция 2.\\ \Large Задачи типизации $\lambda_\rightarrow$. \\Выразительная сила $\lambda_\rightarrow$.\end{center}
\end{frame}

\begin{frame}[fragile]{Основные задачи типизации $\lambda$-исчисления}

Рассмотрим $? \vdash ? : ?$.

		\begin{enumerate}
			\item \emph{Проверка типа:} выполняется ли $\Gamma\vdash M:\sigma$ для контекста $\Gamma\text{, терма }M\text{ и типа }\sigma$\\

{\color{gray}Компиляция в языке программирования, типизированном по Чёрчу. Проверка доказательства.}
			\item \emph{Реконструкция типа:} $?\vdash M:\:?$.

{\color{gray}Компиляция в языке программирования, типизированном по Карри. Это бывает чаще, чем кажется.}

\verb!template <class A, class B>!\\
\verb!auto min(A a, B b) -> decltype(a < b ? a : b) {!\\
\verb!    return (a < b) ? a : b;!\\
\verb!}!

			\item \emph{Обитаемость типа:} $\Gamma\vdash ?:\sigma$.

{\color{gray}Поиск доказательства.}
		\end{enumerate}
Все задачи разрешимы.
\end{frame}

\begin{frame}{Построение системы по терму $M$}
Будем строить систему рекурсией по структуре терма $M$ (предполагаем, что все имена
для связанных переменных уникальны). 
Каждой переменной $x$ сопоставим свежую типовую переменную $\alpha_x$. Также каждой аппликации $P\ Q$ в терме сопоставим
свежую типовую переменную $\beta_{PQ}$. 


%В итоге получим $\langle \mathcal{E}, \sigma\rangle$, где $\mathcal{E}$ ---
%система уравнений, а $\sigma$ --- тип терма $M$. 
По терму $M$ и по всем его подтермам рекурсивно построим пару $\langle \mathcal{E}_M, \sigma_M\rangle$ так:

%Учитывайте недостаток записи: здесь две разные переменные $\beta_{xx}$: $(x\ x)\ (x\ x)$

$$\langle\mathcal{E}_M,\sigma_M\rangle := \left\{\begin{array}{ll}\langle \varnothing, \alpha_x\rangle, & M = x\\
\langle \mathcal{E}_P, \alpha_x\rightarrow\sigma_P\rangle, & M = \lambda x.P\\
\langle \mathcal{E}_P\cup\mathcal{E}_Q\cup\{\sigma_P = \sigma_Q\rightarrow\beta_{PQ}\}, \beta_{PQ}\rangle, 
   & M = P\ Q
\end{array}\right.$$

\begin{thm} Если $S = \mathcal{U}(\mathcal{E}_M)$, то наиболее общим решением задачи типизации будет
$\langle\{ x : S(\alpha_x)\ |\ x \in FV(M) \}, S(\sigma_M)\rangle$\end{thm}
\begin{proof} Индукция по структуре $M$. \end{proof}
\end{frame}

\begin{frame}{Пример вывода типов }
\begin{enumerate}
\item Выберем пример ($M = \lambda f.\lambda x.f\ (f\ x)$) и индуктивно составим систему:
\begin{itemize}
\item Для $f\ x$:\quad $\langle\{\alpha_f = \alpha_x\rightarrow\beta_{fx}\},\; \beta_{fx}\rangle$
\item Для $f\ (f\ x)$:\quad $\langle\{\alpha_f = \alpha_x\rightarrow\beta_{fx}, \alpha_f = \beta_{fx}\rightarrow\beta_{ffx}\}, \beta_{ffx}\rangle$
\item Для $\lambda x.f\ (f\ x)$:\quad $\langle\{\alpha_f = \alpha_x\rightarrow\beta_{fx}, \alpha_f = \beta_{fx}\rightarrow\beta_{ffx}\}, \alpha_x\rightarrow\beta_{ffx}\rangle$
\item Для $\lambda f.\lambda x.f\ (f\ x)$:\quad $\langle\{\alpha_f = \alpha_x\rightarrow\beta_{fx}, \alpha_f = \beta_{fx}\rightarrow\beta_{ffx}\}, \alpha_f\rightarrow\alpha_x\rightarrow\beta_{ffx}\rangle$
\end{itemize}
\item Приводим систему к разрешённой форме:
\begin{itemize}
\item $\alpha_f = \alpha_x\rightarrow\beta_{fx}, \alpha_f = \beta_{fx}\rightarrow\beta_{ffx}$
\item $\alpha_f = \alpha_x\rightarrow\beta_{fx}, \alpha_x\rightarrow\beta_{fx} = \beta_{fx}\rightarrow\beta_{ffx}$, правило $(d)$
\item $\alpha_f = \alpha_x\rightarrow\beta_{fx}, \alpha_x = \beta_{fx}, \beta_{fx} = \beta_{ffx}$, правило $(c)$
\item $\alpha_f = \beta_{fx}\rightarrow\beta_{fx}, \alpha_x = \beta_{fx}, \beta_{fx} = \beta_{ffx}$, правило $(d)$
\item $\alpha_f = \beta_{ffx}\rightarrow\beta_{ffx}, \alpha_x = \beta_{ffx}, \beta_{fx} = \beta_{ffx}$, правило $(d)$
\end{itemize}
\item Строим функцию подстановки:
$S_0(\alpha_f) = \beta_{ffx}\rightarrow\beta_{ffx}, S_0(\alpha_x) = S_0(\beta_{fx}) = \beta_{ffx}$
\item Наиболее общая пара:
$\langle \varnothing, S(\alpha_f\rightarrow\alpha_x\rightarrow\beta_{ffx})\rangle$, то есть
$$\vdash \lambda f.\lambda x.f\ (f\ x) : (\beta_{ffx}\rightarrow\beta_{ffx})\rightarrow\beta_{ffx}\rightarrow\beta_{ffx}$$
\end{enumerate}
\end{frame}

\begin{frame}{Проверка типа}
\begin{enumerate}
\item Задача реконструкции типа находит наиболее общую типизацию.
\item Сведём задачу проверки $\Gamma\vdash M:\sigma$ к задаче реконструкции типа $?\vdash M:?$ и найдём $\langle \Gamma', \sigma' \rangle$.
\item Проверим, является ли $\langle \Gamma, \sigma\rangle$ частным случаем $\langle \Gamma', \sigma' \rangle$.
\end{enumerate}
\end{frame}

\begin{frame}{Обитаемость типа}
\begin{enumerate}
\item Задача поиска $M$, что $\Gamma\vdash M:\sigma$.
\item Эквивалентно поиску доказательства утверждения $\sigma$ в ИИП (разрешимо).
\item По доказательству затем получим его краткую запись в виде терма.
\end{enumerate}
\end{frame}

\begin{frame}{Выразительная сила}
\begin{dfn}
Расширенный полином, где $P(x)$, $P(x,y)$ --- полиномы (выражения, составленные из сложения, умножения, аргументов и натуральных констант),
$c$ --- константа:
$$E(m,n) := \left\{\begin{array}{ll}
c,& m = 0, n = 0\\
P_1(m), & n=0\\
P_2(n), & m=0\\
P_3(m,n), & m>0, n>0
\end{array}\right.$$
\end{dfn}

\begin{thm}
Пусть $\eta = (\alpha\rightarrow\alpha)\rightarrow(\alpha\rightarrow\alpha)$. 
Если $F: \eta\rightarrow\eta\rightarrow\eta$, то найдётся такой расширенный
полином $E(m,n)$, что при всех $m,n \in \mathbb{N}_0$ выполнено $F\ \overline{m}\ \overline{n} =_\beta \overline{E(m,n)}$, 
либо $F\ \overline{m}\ \overline{n} =_\beta \lambda f.f$ при $E(m,n) = 1$.
\end{thm}
\end{frame}

\begin{frame}{Расширение языка: полное ИИВ}
\begin{itemize}
\item Попробуем увеличить выразительную силу, воспользовавшись изоморфизмом Карри-Ховарда. Рассмотрим полное ИИВ.
\item Расширим язык: $$\begin{array}{rcll}\Lambda &::=& x\ |\ (\Lambda\ \Lambda)\ |\ (\lambda x.\Lambda)\ \\
& | & \langle\Lambda,\Lambda\rangle\ |\ (\pi_L\ \Lambda)\ |\ (\pi_R\ \Lambda) & \text{термы для }\&\\
& | & (In_L\ \Lambda)\ |\ (In_R\ \Lambda)\ |\ (Case\ \Lambda\ \Lambda\ \Lambda) & \text{термы для }\vee\\
& | &(Absurd\ \Lambda) & \text{термы для }\bot
\end{array}$$
\end{itemize}

\end{frame}

\begin{frame}{Связки ИИВ не выражаются друг через друга}
\begin{thm}[случай связки $(\rightarrow)$]
Какое бы ни было выражение $\varphi(A,B)$, составленное из связок $(\with)$,$(\vee)$,$\bot$, 
найдётся оценка, что $\llbracket A \rightarrow B\rrbracket \ne \llbracket \varphi(A,B) \rrbracket$\end{thm}

\begin{proof}
Рассмотрим алгебру Гейтинга на $\mathbb{R}$, пусть $\llbracket A\rrbracket = (0,\infty)$ 
и $\llbracket B\rrbracket = \varnothing$.

Заметим, что связки замкнуты относительно множества $\{\varnothing,\mathbb{R},(0,\infty)\}$:

\vspace{0.3cm}
\begin{tabular}{l|l|l}
$(\bot)$ & $(\with)$ & $(\vee)$ \\\hline
& $\mathbb{R} \cap X = X$ & $\mathbb{R} \cup X = \mathbb{R}$\\
$\varnothing$ & $\varnothing\cap X = \varnothing$ & $\varnothing\cup X = X$ \\
& $(0,\infty)\cap X \in \{\varnothing,(0,\infty)\}$ & $(0,\infty)\cup X \in \{(0,\infty),\mathbb{R}\}$
\end{tabular}

\vspace{0.3cm}
Однако, $\llbracket A \rightarrow B \rrbracket = (-\infty,0)$.
\end{proof}

\end{frame}

\begin{frame}{Новые связки требуют отдельных правил}
\begin{itemize}\item Упорядоченная пара в бестиповом лямбда-исчислении.
$$\text{MkPair} ::= \lambda a.\lambda b.\underbrace{\left(\lambda p.p\ a\ b\right)}_{\text{MkPair a b}}
\quad\quad\text{Fst} ::= \lambda p.p\ \text{T}
\quad\quad\text{Snd} ::= \lambda p.p\ \text{F}$$
\item Какой тип у $\text{MkPair a b}$?

$$\text{MkPair a b} = \lambda p^{\alpha\rightarrow\beta\rightarrow\gamma}.p\ a^{\alpha}\ b^{\beta} : (\alpha\rightarrow\beta\rightarrow\gamma)\rightarrow\gamma$$

\item Тип зависит от типа результата $\gamma$: при левой проекции $\alpha = \gamma$ $$\text{Fst} = \lambda p.p^{(\alpha\rightarrow\beta\rightarrow\gamma)\rightarrow\gamma}\ \text{T}^{\alpha\rightarrow\beta\rightarrow\alpha} : \gamma$$

При правой проекции $\beta = \gamma$: $\text{Snd} = \lambda p.p^{(\alpha\rightarrow\beta\rightarrow\gamma)\rightarrow\gamma}\ \text{F}^{\alpha\rightarrow\beta\rightarrow\beta} : \gamma$

\item Из-за невыразимости связок друг через друга в ИИВ никакая формула не сможет типизировать упорядоченную пару. 
Однако, в данном варианте типизации может помочь квантор по $\gamma$ или схема аксиом (правил вывода).
\end{itemize}
\end{frame}

\begin{frame}{Дополнительные правила для расширенного языка}
\begin{enumerate}
	\item Типизация дизъюнкции (алгебраического типа)\[
	\infer{\Gamma \vdash In_L \; A: \varphi \vee \psi}{\Gamma\vdash A: \varphi}
	\quad\quad
	\infer{\Gamma \vdash In_R \; B: \varphi \vee \psi}{\Gamma\vdash B: \psi}
	\]\vspace{-0.5cm}
	\[
	%\quad\quad
	\infer{\Gamma \vdash \text{Case} \; L \; f \; g : \tau}{\Gamma \vdash L: \varphi \vee \psi, \;\;\;\; \Gamma \vdash f : \varphi \to \tau, \;\;\;\; \Gamma \vdash g: \psi \to \tau}
	\]

	\item Типизация конъюнкции (упорядоченной пары)
	\[
	\infer{\Gamma \vdash \langle A, B \rangle : \varphi \text{\&} \psi}{\Gamma\vdash A: \varphi && \Gamma\vdash B: \psi}
	%\]\vspace{-0.5cm}
	%\[
	\quad\quad
	\infer{\Gamma \vdash \text{Fst}\ P : \varphi}{\Gamma \vdash P: \varphi \text{\&} \psi}
	\quad\quad
	\infer{\Gamma \vdash \text{Snd}\ P : \psi}{\Gamma \vdash P: \varphi \text{\&} \psi}
	\]

	\item Типизация лжи
	\[
	\infer{\Gamma \vdash \text{Absurd}\ A : \varphi}{\Gamma \vdash A : \bot}
	\]
\end{enumerate}
\end{frame}

\begin{frame}{Нормализуемость}
\begin{dfn}
\begin{itemize}
\item Терм $A$ назовём слабо нормализуемым, если существует последовательность редукций,
приводящих его в нормальную форму.
\item Терм $A$ назовём сильно нормализуемым, если не существует бесконечной последовательности
его редукций.
\item Исчисление назовём сильно нормализуемым, если любой его терм сильно нормализуем.
\end{itemize}
\end{dfn}

\vspace{-0.3cm}
\begin{thm}Бестиповое лямбда-исчисление не является сильно нормализуемым\end{thm}
\vspace{-0.3cm}
\begin{proof}$\Omega \rightarrow_\beta \Omega$\end{proof}

\begin{thm}Просто типизированное лямбда-исчисление является сильно нормализуемым\end{thm}
\end{frame}

\begin{comment}
\begin{frame}{$SN_\sigma$: сильно нормализуемые термы типа $\sigma$}
\begin{dfn} Будем записывать $SN_\alpha(T)$, если \begin{enumerate}
\item $\vdash T : \alpha$
\item $T$ сильно нормализуем.
\end{enumerate}\vspace{0.5cm}

$SN_{\sigma\rightarrow\tau}(T)$, если \begin{enumerate}
\item $\vdash T : \sigma\rightarrow\tau$
\item $T$ сильно нормализуем
\item Если $SN_\sigma(P)$, то $SN_\tau(T\ P)$.
Неформальное пояснение: функция гарантированно заканчивает работу на любых сильно нормализуемых аргументах.
\end{enumerate}

\end{dfn}
\end{frame}

\begin{frame}[fragile]{Пример 1}

{\color{gray}$SN_\zeta(T)$, если:
$\vdash T:\zeta$, $T$ сильно нормализуем, при $\zeta=\sigma\rightarrow\tau$ и $SN_\sigma(P)$
выполнено $SN_\tau(T\ P)$
}

\begin{exm}
$SN_{\alpha\rightarrow\alpha}(\lambda x.x)$:
\begin{enumerate}
\item $\vdash \lambda x.x : \alpha\rightarrow\alpha$
\item $\lambda x.x$ находится в нормальной форме
\item Каждой цепочке редукций $I\ P$ соответствует цепочка редукций $P$ длины $k-1$ или $k$
(редукции внутри $P$ оставляем, редукцию $I\ Q \rightarrow_\beta Q$ пропускаем).
Отсюда, если $SN_\alpha(P)$, то $I\ P$ сильно нормализуемо и $\vdash I\ P : \alpha$.
\end{enumerate}
\end{exm}
\end{frame}

\begin{frame}[fragile]{Пример 2}

\begin{minipage}{10cm}{\color{gray}$SN_\zeta(T)$, если:\\
 (1) $\vdash T:\zeta$, (2) $T$ сильно нормализуем, \\(3) при $\zeta=\sigma\rightarrow\tau$ и $SN_\sigma(P)$
выполнено $SN_\tau(T\ P)$}
\end{minipage}

\begin{exm}
Важность третьего условия: не обязательно результат подстановки сильно нормализуемого терма 
в сильно нормализуемый терм сильно нормализуем.

\begin{itemize}
\item $\lambda x.x\ x$ сильно нормализуем.
\item Бывает так: $(\lambda x.x\ x)\ I \rightarrow_\beta I\ I \rightarrow_\beta I$
\item Но бывает и так: $(\lambda x.x\ x)\ (\lambda x.x\ x)$.
\end{itemize}
\end{exm}

\end{frame}

\begin{frame}
\begin{thm}Если $\vdash S:\sigma$ и $S\rightarrow_\beta T$, то $SN_\sigma(S)$ тогда и только тогда, когда $SN_\sigma(T)$.\end{thm}
\begin{proof}Лемма о редукции типа, индукция по структуре $\sigma$

$(\lambda x.P) Q -> P [x:=Q]$
Если $\sigma = \alpha$, то $SN_\alpha(T)$ означает, что 

\end{proof}
\end{frame}

\begin{frame}{Применение типизированного терма к $SN$-аргументам само $SN$}
Если $\{x_i : \sigma_i\} \vdash Q : \tau$ и $SN_{\sigma_i}(P_i)$, то $SN_\tau(Q[x_1 := P_1,\dots,x_n := P_n])$.
Индукция по структуре доказательства $x_i : \sigma_i \vdash Q:\tau$.
\begin{enumerate}
\item $$\infer{\vdash x_k : \sigma_k}{\{x_i : \sigma_i\}}$$
\item $$\infer{\Gamma\vdash M\ N : \psi}{\Gamma\vdash M : \varphi\rightarrow\psi\quad\quad N : \psi}$$
По индукции, $SN_{\varphi\rightarrow\psi}(M[{x_i := P_i}])$. Также, по определению имеем $SN_{\varphi\rightarrow\psi}$.
Отсюда $SN_\psi(M[{x_i := P_i}]\ N[{x_i := P_i}])$
\end{enumerate}
\end{frame}

\begin{frame}{Типизированный терм является сильно нормализованым}
\begin{thm}Если $\vdash T:\tau$, то $T сильно нормализуем.$\end{thm}
\begin{proof}По вырожденному случаю леммы о применении типизированного терма $SN_\tau(T[])$ (при пустом списке замен).
Тогда по определению $SN_\tau(T)$ имеем сильную нормализуемость $T$.\end{proof}
\end{frame}
\end{comment}

\begin{frame}{Сильно нормализуемые множества}
\begin{dfn}SN --- множество всех сильно нормализуемых лямбда-термов.

Насыщенное множество $\mathcal{X} \subseteq SN$ --- такое, что: \begin{enumerate}
\item для любых $n \ge 0$ и $M_1, \dots, M_n \in SN$ $$x\ M_1 \dots M_n \in \mathcal{X}$$
\item для любых $n \ge 1$, $M_1, \dots, M_n \in SN$ и $N \in \Lambda$ $$N[x := M_1]\ M_2 \dots M_n \in \mathcal{X}\text{ влечёт }(\lambda x.N)\ M_1\ M_2\dots M_n \in \mathcal{X}$$
\end{enumerate}
\end{dfn}

\vspace{-0.7cm}
\begin{lmm}
$SN$ --- насыщенное. 
\end{lmm}
Интересен пункт 2: если $N[x := M_1]\ M_2\dots M_n \in SN$, то $\underline{(\lambda x.N)\ M_1}\ M_2\dots M_n \in SN$. Подстановка подчёркнутого 
возвращает к редукции посылки, бесконечная <<локальная>> подстановка может быть повторена с посылкой.
\end{frame}

\begin{frame}{}
\begin{dfn}
Если $\mathcal{A},\mathcal{B} \subseteq \Lambda$, то $\mathcal{A} \rightarrow \mathcal{B} = \{ X \in \Lambda\ |\ \forall Y \in \mathcal{A}\ .\ X\ Y \in \mathcal{B}\}$
\end{dfn}

\begin{exm}
$\{ \lambda x.\lambda y.x \} \rightarrow \{ X\ |\ X =_\beta \lambda x.\lambda y.y \} = \{ Not, \lambda t.F, Xor\ T, \dots \}$
\end{exm}

\begin{dfn}
$$\llbracket \sigma \rrbracket = \left\{\begin{array}{ll}SN, & \sigma = \alpha\\
  \llbracket \tau_1 \rrbracket \rightarrow \llbracket \tau_2 \rrbracket, & \sigma = \tau_1 \rightarrow \tau_2\end{array}\right.$$
\end{dfn}

\begin{lmm}
Если $\mathcal{A}, \mathcal{B}$ насыщены, то $\mathcal{A}\rightarrow\mathcal{B}$ насыщено\\
$\llbracket\sigma\rrbracket$ насыщено.
\end{lmm}

\begin{lmm}$\llbracket\sigma\rrbracket\subseteq SN$\end{lmm}

\end{frame}

\begin{frame}{Оценка}
\begin{dfn}
Оценка $\rho: \mathcal{V} \rightarrow \Lambda$ --- отображение переменных в лямбда-термы.

$M_\rho := M[x_1 := \rho(x_1), \dots, x_n := \rho(x_n)]$, где $x_i$ --- все свободные переменные $M$.

Будем писать $\rho \models M:\sigma$, если $M_\rho \in \llbracket\sigma\rrbracket$.
Будем писать $\rho \models \Gamma$, если $\rho(x) \in \llbracket\sigma\rrbracket$ для всех $x : \sigma \in \Gamma$.

$\Gamma \models M:\sigma$, если для любой оценки $\rho$ из $\rho\models\Gamma$ следует $\rho \models M:\sigma$.
\end{dfn}

\begin{thm}$\Gamma\vdash M:\sigma$ влечёт $\Gamma\models M:\sigma$.
\end{thm}

Доказательство индукцией по структуре вывода $\Gamma\vdash M:\sigma$ со следующим разбором случаев.

\end{frame}

\begin{frame}{Аксиома}
Вывод имеет вид:

$$\infer{\Gamma, x : \sigma \vdash x:\sigma}{}$$

Фиксируем $\rho \models \Gamma\cup\{x : \sigma\}$, тогда
$x_\rho = \rho(x)\in\llbracket\sigma\rrbracket$

Отсюда $\Gamma, x : \sigma \models x : \sigma$
\end{frame}

\begin{frame}{Применение}
Вывод имеет вид:
$$\infer{\Gamma\vdash M\ N:\tau}{\Gamma\vdash M:\sigma\rightarrow\tau\quad\quad\Gamma\vdash N:\sigma}$$
Фиксируем $\rho \models \Gamma$. По индукционному предположению, $\Gamma\models M:\sigma\rightarrow\tau$ и
$\Gamma\models N:\sigma$, так что $\rho\models M:\sigma\rightarrow\tau$ и $\rho\models N:\sigma$, что означает, что
$M_\rho \in \llbracket\sigma\rrbracket\rightarrow\llbracket\tau\rrbracket$ и $N_\rho \in \llbracket\sigma\rrbracket$.
Тогда $(M\ N)_\rho = M_\rho\ N_\rho \in \llbracket\tau\rrbracket$.
\end{frame}
\begin{frame}{Абстракция}
Вывод имеет вид:
$$\infer[x \notin FV(\Gamma)]{\Gamma\vdash\lambda x.M : \sigma\rightarrow\tau}{\Gamma,x:\sigma\vdash M:\tau}$$

Пусть $\rho\models\Gamma$.
Чтобы показать $(\lambda x.M)_\rho \in \llbracket\sigma\rightarrow\tau\rrbracket$, надо для всех $N \in \llbracket\sigma\rrbracket$
показать $(\lambda x.M)_\rho\ N \in \llbracket\tau\rrbracket$.

Фиксируем $N \in \llbracket\sigma\rrbracket$. Тогда $\rho^{x := N}\models\Gamma,x:\sigma$. 
По индукционному предположению, $\Gamma,x:\sigma\models M:\tau$, так что $\rho^{x := N}\models M:\tau$ (по определению $\models$).
То есть, $M_{\rho^{x := N}} \in \llbracket\tau\rrbracket$. Произведём редукцию:
$$(\lambda x.M)_\rho N = (\lambda x.M)^{y_1 := \rho(y_1), \dots, y_n := \rho(y_n)}\ N \rightarrow_\beta M^{y_1 := \rho(y_1), \dots, y_n := \rho(y_n), x:=N}
 = M_{\rho^{x := N}}$$

Заметим, $N \in \llbracket \sigma \rrbracket \subseteq SN$ и $M_{\rho^{x := N}} \in \llbracket\tau\rrbracket$.
Заметим ещё, что $M_{\rho^{x:=N}} = M_\rho[x := N]$.
По определению насыщенного множества из $M_\rho[x := N] \in \llbracket\tau\rrbracket$ следует требуемое $(\lambda x.M)_\rho\ N \in \llbracket\tau\rrbracket$.
\end{frame}
\begin{frame}{Основная теорема}
\begin{thm}$\Gamma\vdash M:\sigma$ влечёт $M \in SN$\end{thm}
\begin{proof}
По предыдущей теореме, $\Gamma\models M:\sigma$. Построим <<тождественную>> оценку, $\rho(x) = x$ для всех $x:\tau \in \Gamma$.

Рассмотрим каждый $x : \tau$ из контекста.
По лемме выше, $\llbracket\tau\rrbracket$ насыщенное. По определению насыщенного, $x \in \llbracket\tau\rrbracket$.
Поэтому $\rho\models\Gamma$.

Поскольку $\Gamma\models M:\sigma$, то $M = M_\rho \in \llbracket \sigma \rrbracket$. А по лемме выше, $\llbracket \sigma \rrbracket \subseteq SN$.
\end{proof}
\end{frame}

\begin{frame}{О свойстве сильной нормализуемости}

Правило сечения в $S_\infty$ (без одной боковой формулы):
$$\infer{\sigma}{\sigma\vee\neg\beta\quad\quad\beta}$$
Или перепишем в привычной грамматике (подобно Modus Ponens):
$$\infer{\sigma}{\beta\rightarrow\sigma\quad\quad\beta}$$
И заметим нечто похожее в просто-типизированном лямбда-исчислении:
$$\infer[\beta-\text{редекс}]{(\lambda x.P)\ Q:\sigma}{(\lambda x.P) : \tau\rightarrow\sigma\quad\quad Q:\tau}$$

Поэтому добавим пункты к изоморфизму Карри-Ховарда:
\begin{center}\begin{tabular}{l|l}
Логика & $\lambda_\rightarrow$ \\\hline
Правило сечения, M.P. & Бета-редекс\\
Устранение сечения & Бета-редукция\\
Теорема об устранении сечений & Нормализуемость
\end{tabular}\end{center}

\end{frame}

\end{document}
