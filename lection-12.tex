\documentclass[aspectratio=169,dvipsnames,usenames]{beamer}
\usepackage{graphicx}
\usepackage{stmaryrd}
\usepackage[T1,T2A]{fontenc}
\usepackage[utf8]{inputenc}
\usepackage[english,russian]{babel}
\usepackage{amsmath}
\usepackage{amsfonts}
\usepackage{amssymb}
\usepackage{makeidx}
\usepackage{verbatim}
\usepackage{amsthm}
%\usepackage{bnf}
\usepackage{cmll}
\usepackage{tikz}
\usepackage{enumerate}
\usepackage{mathtext}
\usepackage{mathtools}
\usepackage{mathabx}
%\usepackage[left=2cm,right=2cm,top=2cm,bottom=2cm,bindingoffset=0cm]{geometry}
\usepackage{proof}
%\usepackage{paracol}
%\usepackage{enumitem}
\usepackage{xcolor}
\usepackage{colortbl}
%\usepackage{minted}
%\usepackage{hyperref}
\setbeamertemplate{navigation symbols}{}
\usetikzlibrary{graphs}
\usetikzlibrary{graphs.standard}
\usetikzlibrary{automata,positioning}
\usepackage{float}

\begin{document}

%\theoremstyle{dfn}
\newtheorem{dfn}{Определение}[section]
\newtheorem{nte}{Замечание}[section]

\newtheorem{axiom}{Аксиома}[section]
\newtheorem{thm}{Теорема}[section]
\newtheorem{lmm}[theorem]{Лемма}
\newtheorem{statement}{Утверждение}[section]
\newtheorem{oun_paragraph}{Пункт}[section]
\newtheorem{cons}{Следствие}[section]
\newtheorem*{exm}{Пример}

\newcommand{\comb}[1]{\operatorname{\mathcal{#1}}}
\newcommand{\func}[1]{\operatorname{#1}}
\newcommand{\reduction}[1]{{\color{OrangeRed}#1}}
\newcommand{\set}[1]{\left\{#1\right\}}

\def\from#1{\par \parbox{0.7\textwidth}{\par \hfill\raggedleft \it #1}} 

\begin{frame}{}
\begin{center}
{\LARGE Проверка на модели (model checking)}
\end{center}
\end{frame}

\begin{frame}{Общее введение}
\begin{itemize}
\item Проверить многопоточный алгоритм/протокол --- очень сложно.
\item Перебрать, двоичные деревья --- комбинаторный взрыв. Нужен более содержательный подход.
\item Проверка на моделях --- тоже по сути перебор, только очень хорошо оптимизированный.
\item Первоначальная идея: Сlarke E., Emerson E., Sistla A., Automatic verification of finite-state con-
                                                            current systems using temporal logic specifications 
%ACM Trans. Prog. Lang. and Systems, vol. 8, num. 2, p. 244–263, 
1986.
\end{itemize}
\end{frame}

\begin{frame}{Линейная темпоральная логика: операторы}
\begin{itemize}
\item Вариант моделей Крипке для модальной логики (интуиция из интуиционистской логики подходит).
\item Миры выстроены в линейном порядке.
\item В моделях Крипке для ИИВ (неявная) зависимость от миров была у импликации и отрицания.
В ЛТЛ зависиость явная, и выражается с помощью следующих дополнительных связок:
\begin{enumerate}
\item $G(\alpha)$ или $\Box\alpha$: утверждение $\alpha$ выполнено в любой момент (начиная с текущего).
\item $P(\alpha)$ или $\bigcirc\alpha$: утверждение $\alpha$ выполнено в следующий момент.
\item $E(\alpha)$ или $\Diamond\alpha$: утверждение $\alpha$ неизбежно выполнено в будущем, в какой-то момент (начиная с текущего).
\item $U(\alpha,\beta)$ или $\alpha\mathcal{U}\beta$: утверждение $\alpha$ истинно, пока $\beta$ не станет истинным,
после чего $\alpha$ может быть любым.
\end{enumerate}
\end{itemize}
\end{frame}

\begin{frame}{Примеры выражений}
\end{frame}

\begin{frame}{Постановка задачи}
\begin{dfn}Системой переходов назовём граф состояний, в котором каждое состояние отражает
содержимое памяти компьютера, а переходы соответствуют инструкциям (операциям), выполняемым
компьютером\end{dfn}

Хотим научиться проверять, выполнено ли $\varphi$ при всех возможных вариантах выполнения программы,
то есть при всех возможных путях в системе переходов $TS$:

$$TS \models \varphi$$

Можем понимать систему переходов как модель для логического исчисления.
\end{frame}

\begin{frame}{Автоматы Бюхи}
\begin{itemize}
\item Автоматы Бюхи отличаются от конечного автомата правилом проверки 
допустимости строки:
бесконечная строка $\alpha$ допускается конечным автоматом Бюхи, если в процессе применения
автомата к ней допускающее состояние будет посещено бесконечное количество раз.

\item
Рассмотрим автомат с двумя состояниями (0,1) и полным графом переходов.

Строка $(ab)^\omega$ будет принята, строка $a(b^\omega)$ --- нет.

\item Можно рассматривать недетерминированные автоматы Бюхи.
\item Обобщённые недетерменированные автоматы Бюхи --- такие, где есть список допускающих
состояний, каждое из которых должно быть посещено бесконечное количество раз.
\end{itemize}
\end{frame}

\begin{frame}{Представление моделей ЛТЛ как множества слов}
Рассмотрим следующую грамматику для формул: $\varphi ::= \top\ |\ a\ |\ \varphi\with\varphi\ |\ \neg\varphi\ |\ \bigcirc\varphi\ |\ \varphi\mathcal{U}\varphi$

Тогда:
$$W(\varphi) = \{\sigma\in (\mathcal{P}(a))^\omega\ |\ \sigma\models\varphi \}$$
На строке $\sigma = S_0S_1S_2\dots$ (каждый $S_i \subseteq \mathcal{P}(a)$) истинность задаётся так:
$$\begin{array}{ll}
\sigma\models\top & \text{всегда}\\
\sigma\models a & a \in S_0\\
\sigma\models \varphi_1 \with \varphi_2 & \sigma\models\varphi_1 \text{ и } \sigma\models\varphi_2\\
\sigma\models\neg\varphi & \sigma\not\models\varphi\\
\sigma\models\bigcirc\varphi & \sigma[1\dots]\models\varphi\\
\sigma\models\varphi_1\mathcal{U}\varphi_2 & \text{ существует }j.\sigma[j\dots]\models\varphi_2 \text{ и } \sigma[i\dots]\models\varphi_1, i < j
\end{array}$$
\end{frame}

\begin{frame}{Разрешимость задачи $TS \models \varphi$ в ЛТЛ}
\begin{thm}Если система переходов конечна и не содержит терминальных состояний, то
существует алгоритм, разрешающий данную задачу для любого $\varphi$. В случае отрицательного
ответа алгоритм строит контрпример.
\end{thm}

\begin{proof}Построим недетерменированный автомат Бюхи по системе переходов для формулы $\neg\varphi$
(принимает строку тогда и только тогда, когда она удовлетворяет формуле), затем по автомату построим
удовлетворяющую строку, если она существует.
\end{proof}
\end{frame}

\begin{frame}{Пример реализации: SPIN}
\begin{itemize}
\item Один из первых инструментов (1991 год).
\item Используется специальный язык для описания алгоритмов/протоколов (Promela).
\item Язык позволяет формализовать параллельные вычисления.
\item Программа может быть вычислена в разных окружениях (например, случайное исполнение).
\item Также, к программе могут быть добавлены условия, которые либо будут доказаны --- либо
будет найден контрпример (контрпример будет предъявлен).
\end{itemize}
\end{frame}

\begin{frame}[fragile]{Пример программы на Promela: каковы возможные значения n?}
\begin{verbatim}
byte n = 0, finish = 0;
active [2] proctype P() {
    byte register, counter = 0;
    do :: counter = 10 -> break
       :: else ->
             register = n;
             register++;
             n = register;
             counter++
    od;
    finish++
}
active proctype WaitForFinish() {
    finish == 2;
    printf(“n = %d\n”, n)
}
\end{verbatim}
\end{frame}

%\begin{frame}{Сколько возможно ответов?}
%\begin{itemize}
%\pause\item На самом деле два.
%\end{itemize}
%\end{frame}
%\begin{frame}{}
%\end{frame}

\end{document}
