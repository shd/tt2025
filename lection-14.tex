\documentclass[aspectratio=169,dvipsnames,usenames]{beamer}
\usepackage{graphicx}
\usepackage{stmaryrd}
\usepackage[T1,T2A]{fontenc}
\usepackage[utf8]{inputenc}
\usepackage[english,russian]{babel}
\usepackage{amsmath}
\usepackage{amsfonts}
\usepackage{amssymb}
\usepackage{makeidx}
\usepackage{verbatim}
\usepackage{amsthm}
%\usepackage{bnf}
\usepackage{cmll}
\usepackage{tikz}
\usepackage{enumerate}
\usepackage{mathtext}
\usepackage{mathtools}
\usepackage{mathabx}
%\usepackage[left=2cm,right=2cm,top=2cm,bottom=2cm,bindingoffset=0cm]{geometry}
\usepackage{proof}
%\usepackage{paracol}
%\usepackage{enumitem}
\usepackage{xcolor}
\usepackage{colortbl}
%\usepackage{minted}
%\usepackage{hyperref}
\setbeamertemplate{navigation symbols}{}
\usetikzlibrary{graphs}
\usetikzlibrary{graphs.standard}
\usetikzlibrary{automata,positioning}
\usepackage{float}

\begin{document}

%\theoremstyle{dfn}
\newtheorem{dfn}{Определение}[section]
\newtheorem{nte}{Замечание}[section]

\newtheorem{axiom}{Аксиома}[section]
\newtheorem{thm}{Теорема}[section]
\newtheorem{lmm}[theorem]{Лемма}
\newtheorem{statement}{Утверждение}[section]
\newtheorem{oun_paragraph}{Пункт}[section]
\newtheorem{cons}{Следствие}[section]
\newtheorem*{exm}{Пример}

\newcommand{\comb}[1]{\operatorname{\mathcal{#1}}}
\newcommand{\func}[1]{\operatorname{#1}}
\newcommand{\reduction}[1]{{\color{OrangeRed}#1}}
\newcommand{\set}[1]{\left\{#1\right\}}

\def\from#1{\par \parbox{0.7\textwidth}{\par \hfill\raggedleft \it #1}} 

\begin{frame}{}
\begin{center}
{\LARGE Неразрешимость системы F}
\end{center}
\end{frame}

\begin{frame}{Разрешимость типизации $\lambda_\rightarrow$}
\begin{itemize}
\item Алгебраические термы: $T ::= V\ |\ f\ T\ \dots\ T$,
\item Уравнение в алгебраических термах $\sigma = \sigma'$
\item Подстановка переменной: функция $S_0: V \rightarrow T$, тожественная почти везде.
\item Подстановка: $S: T \rightarrow T$, что $S(v) = S_0(v)$ и 
$S(f\ \theta_1\ \dots\ \theta_n) = f\ S(\theta_1)\ \dots\ S(\theta_n)$
\item Решение задачи унификации --- такая $S$, что $S(\sigma) = S(\sigma')$.
\item Задача типизации: по лямбда-выражению строим уравнение, находим решение, по решению строим тип.
\end{itemize}
\end{frame}

\begin{frame}{Система F}
$$\infer[\text{(Акс.)}]{\Gamma,x:\sigma\vdash x:\sigma}{}$$
$$\infer[\text{(Прим.)}]{\Gamma\vdash(M\ N):\tau}{\Gamma\vdash M:\sigma\rightarrow\tau\quad \Gamma\vdash N:\sigma}$$
$$\infer[\text{(Абстр.)}]{\Gamma\vdash(\lambda x.M):\sigma\rightarrow\tau}{\Gamma,x:\sigma\vdash M:\tau}$$
$$\infer[\text{(Спец.)}]{\Gamma\vdash M:(\sigma[\alpha:=\tau])}{\Gamma\vdash M:\forall\alpha.\sigma}$$
$$\infer[\text{(Обобщ.)}]{\Gamma\vdash M:\forall\alpha.\sigma}{\Gamma\vdash M:\sigma}$$

\begin{dfn}
Вывод --- записанное в виде списка дерево. 
То есть, конечная последовательность формул (секвентов) $\Delta_1,\dots,\Delta_n$ 
(вида $\Gamma\vdash M:\sigma$), где каждый $\Delta_i$ --- результат применения 
какого-то из правил к посылкам $\Delta_j$ (и возможно $\Delta_k$), где $j,k < i$.
\end{dfn}
\end{frame}

\begin{frame}{Специализация перед обобщением}
\begin{dfn}
Пусть для доказательства $\Delta_i$ определена функция $c(i)$
(<<номер заключения для посылки $i$>>), при этом все правила в выводе
имеют вид $$\infer[(\text{также }c(i)=c(j))]{\Delta_{c(i)}}{\Delta_i\quad\quad\Delta_j}\quad\quad\infer{\Delta_{c(i)}}{\Delta_i}$$

В доказательстве \emph{специализации идут перед обобщениями}, {\bfseries если для любой}
подпоследовательности $\Delta_{m_1},\dots,\Delta_{m_j}$, такой,
что:\begin{enumerate}
\item $m_i = c(m_{i-1})$;
\item $\Delta_{m_1}$ --- заключение аксиомы, правил применения и абстракции;
\item $\Delta_{m_2},\dots,\Delta_{m_j}$ --- заключения правил обобщения и специализации;
\end{enumerate}

{\bfseries выполнено:} существует $1 \leq k \leq j$, что $\Delta_{m_2},\dots,\Delta_{m_k}$ --- 
заключения правил специализации, а $\Delta_{m_{k+1}},\dots,\Delta_{m_j}$ --- правил обобщения.
\end{dfn}
\end{frame}

\begin{frame}{Полуунификация}
Язык термов: $\mathcal{T} ::= V | \mathcal{T}\rightarrow\mathcal{T}$.

\begin{dfn}Решением задачи полуунификации для $\{\tau_1 \leq \mu_1, \tau_2 \leq \mu_2\}$
при $\tau_1,\tau_2,\mu_1,\mu_2\in\mathcal{T}$ будет
$S: \mathcal{T}\rightarrow\mathcal{T}$, что для неё существуют $S_1$ и $S_2$, что $S_1(S(\tau_1)) = S(\mu_1)$ и 
$S_2(S(\tau_2)) = S(\mu_2)$.\end{dfn}

\begin{thm}Задача полуунификации для $\{\tau_1 \leq \mu_1, \tau_2 \leq \mu_2\}$ неразрешима\end{thm}
Без доказательства.
\end{frame}

\begin{frame}{Сведение полуунификации к проверке типов}
\begin{thm}Разрешимость задачи проверки типов для $F$ эквивалентна разрешимости
полуунификции для $\{\tau_1 \leq \mu_1, \tau_2 \leq \mu_2\}$ при $\tau_1,\tau_2,\mu_1,\mu_2\in\mathcal{T}$.
\end{thm}
\begin{proof}
Пусть даны $\tau_1,\tau_2,\mu_1,\mu_2$ со свободными переменными $\alpha_1,\dots,\alpha_n$.
Рассмотрим контекст:
$$\Gamma := \{ b:\forall\beta\gamma.(\gamma\rightarrow\gamma)\rightarrow\beta,\ \ 
             c:\forall\alpha_1\dots\alpha_n\delta_1\delta_2.(\mu_1\rightarrow\delta_1)\rightarrow(\delta_2\rightarrow\mu_2)\rightarrow(\tau_1\rightarrow\tau_2) \}$$
И построим формулу:
$$\Gamma\vdash (b\ (\lambda x.c\ x\ x)) : \beta$$

\begin{itemize}
\item Решение задачи полуунификации даёт доказуемость формулы в $F$.
\item Наличие доказательства формулы в $F$ позволяет решить задачу полуунификации.
\end{itemize}
\end{proof}
\end{frame}

\begin{frame}{По решению строим доказательство}
Предположим, есть $S, S_1, S_2$, что $S_1(S(\tau_1))=S(\mu_1)$ и $S_2(S(\tau_2))=S(\mu_2)$.
Рассмотрим $\Gamma' := \Gamma,\ x:\forall.S(\tau_1)\rightarrow S(\tau_2)$.

{\color{gray} $\Gamma := \{ b:\forall\beta\gamma.(\gamma\rightarrow\gamma)\rightarrow\beta,\ \ 
             c:\forall\alpha_1\dots\alpha_n\delta_1\delta_2.(\mu_1\rightarrow\delta_1)\rightarrow(\delta_2\rightarrow\mu_2)\rightarrow(\tau_1\rightarrow\tau_2) \}$}

\vspace{0.1cm}
Тогда легко построить вывод следующих формул, применив правила подстановки:
\begin{tabular}{ll}
(1) & $\Gamma' \vdash c : (S(\mu_1)\rightarrow S_1(S(\tau_2)))\rightarrow (S_2(S(\tau_1))\rightarrow S(\mu_2))
                    \rightarrow (S(\tau_1)\rightarrow S(\tau_2))$\\
(2) & $\Gamma' \vdash x : S_1(S(\tau_1)) \rightarrow S_1(S(\tau_2))$\\
(3) & $\Gamma' \vdash x : S_2(S(\tau_1)) \rightarrow S_2(S(\tau_2))$
\end{tabular}

\vspace{0.1cm}
Воспользуемся полуунификацией: $S(\mu_1)=S_1(S(\tau_1))$ и $S(\mu_2)=S_2(S(\tau_2))$

\begin{tabular}{ll}
(4) & $\Gamma' \vdash cx : (S_2(S(\tau_1))\rightarrow S(\mu_2)) \rightarrow (S(\tau_1)\rightarrow S(\tau_2))$\\
(5) & $\Gamma' \vdash cxx : S(\tau_1)\rightarrow S(\tau_2)$
\end{tabular}

\vspace{0.1cm}
$\beta$ --- связанная, потому можем гарантировать $S(\beta)=\beta$, $S_1(\beta)=\beta$, $S_2(\beta)=\beta$.

\begin{tabular}{ll}
(6) & $\Gamma' \vdash cxx : \forall . S(\tau_1) \rightarrow S(\tau_2)$\\
(7) & $\Gamma \vdash \lambda x.cxx : (\forall . S(\tau_1)\rightarrow S(\tau_2)) \rightarrow (\forall.S(\tau_1)\rightarrow S(\tau_2))$\\
(8) & $\Gamma \vdash b : ((\forall . S(\tau_1)\rightarrow S(\tau_2)) \rightarrow (\forall.S(\tau_1)\rightarrow S(\tau_2))) \rightarrow \beta$\\
(9) & $\Gamma \vdash b\ (\lambda x.cxx) : \beta$
\end{tabular}

\end{frame}

\begin{frame}{По доказательству строим решение}

{\color{gray} $\Gamma := \{ b:\forall\beta\gamma.(\gamma\rightarrow\gamma)\rightarrow\beta,\ \ 
             c:\forall\alpha_1\dots\alpha_n\delta_1\delta_2.(\mu_1\rightarrow\delta_1)\rightarrow(\delta_2\rightarrow\mu_2)\rightarrow(\tau_1\rightarrow\tau_2) \}$}
Пусть доказуемо $\Gamma\vdash b\ (\lambda x.cxx) : \beta$. В формуле нет редексов, 
потому каждой конструкции должно соответствовать применение соответствующего правила.

\vspace{0.1cm}
Ветка, определяющая тип $\lambda x.cxx$:
\begin{tabular}{l}
$\Gamma,x:\sigma \vdash c : (T(\mu_1)\rightarrow T(\delta_1))\rightarrow(T(\delta_2)\rightarrow T(\mu_2))\rightarrow(T(\tau_1)\rightarrow T(\tau_2))$\\
$\Gamma,x:\sigma \vdash x : T(\mu_1)\rightarrow T(\delta_1)$\\
$\Gamma,x:\sigma \vdash cx : (T(\delta_2)\rightarrow T(\mu_2))\rightarrow(T(\tau_1)\rightarrow T(\tau_2))$\\
$\Gamma,x:\sigma \vdash x : T(\delta_2)\rightarrow T(\mu_2)$\\
$\Gamma,x:\sigma \vdash cxx : T(\tau_1)\rightarrow T(\tau_2)$\\
$\Gamma,x:\sigma \vdash cxx : \forall.T(\tau_1)\rightarrow T(\tau_2)$\\
$\Gamma\vdash \lambda x.cxx : \sigma \rightarrow \forall.T(\tau_1)\rightarrow T(\tau_2)$
\end{tabular}

\vspace{0.1cm}
Также, вывод типа $b$ должен оканчиваться на такую формулу:
\begin{tabular}{l}
$\Gamma\vdash b : (\varphi\rightarrow\varphi)\rightarrow\beta$
\end{tabular}

\vspace{0.1cm}
Типизация $b(\lambda x.cxx)$ возможна только при помощи правила для применения:
\begin{tabular}{l}
$\Gamma\vdash b(\lambda x.cxx):\beta$
\end{tabular}
\end{frame}

\begin{frame}{Анализ доказательства}
Правило для вывода типа $b(\lambda x.cxx)$
$$
\infer{\Gamma\vdash b(\lambda x.cxx):\beta}{\Gamma\vdash b : (\varphi\rightarrow\varphi)\rightarrow\beta\quad\quad\Gamma\vdash \lambda x.cxx : \sigma \rightarrow \forall.T(\tau_1)\rightarrow T(\tau_2)}
$$
даёт $\varphi=\sigma=\forall.T(\tau_1)\rightarrow T(\tau_2)$

\vspace{0.1cm}
Анализируя же правила типизации $x$:

$$\infer{\Gamma,x:\sigma \vdash x : T(\mu_1)\rightarrow T(\delta_1)}{}\quad
\infer{\Gamma,x:\sigma \vdash x : T(\delta_2)\rightarrow T(\mu_2)}{}$$


получим четыре равенства:

$T_1(T(\tau_1))=T(\mu_1), T_2(T(\tau_1))=T(\delta_2), 
  T_1(T(\tau_2))=T(\delta_1), T_2(T(\tau_2))=T(\mu_2)$

%И закончим построение подстановки путём определения функции стирания:

$$|\sigma|=\left\{\begin{array}{ll}\alpha,&\sigma=\alpha\\
    |\tau_1|\rightarrow|\tau_2|,&\sigma=\tau_1\rightarrow\tau_2\\
    |\tau|,&\sigma=\forall \alpha.\tau\end{array}\right.$$

$S_k(\alpha_k) = |T_k(\alpha_k)|$, и $S_k(\varepsilon)=\varepsilon$ для других переменных.

\end{frame}

\begin{frame}{Дополнительные факты}
\begin{itemize}
\item Задача проверки типов для $F$ эквивалентна задаче нахождения типов.
\item Другие методы доказательства: например, унификация второго порядка.
\end{itemize}
\end{frame}

\end{document}
